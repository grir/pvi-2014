\documentclass[10pt,a4paper]{article}
\usepackage[utf8]{inputenc}
%\usepackage[lithuanian]{babel}
\usepackage{amsmath}
\usepackage{amsfonts}
\usepackage{amssymb}
\usepackage{makeidx}
\usepackage{graphicx}
\usepackage[left=2cm,right=2cm,top=2cm,bottom=2cm]{geometry}
\begin{document}
\begin{enumerate}
\item 
\[
 \int_0^1 (x+1)^{\alpha x}\,\mathrm{d}x, \text{čia } \alpha = 0; 0,01; \cdots; 1
\]

\item 
\[
 \int_0^1 (\alpha x+1)^{x}\,\mathrm{d}x, \text{čia } \alpha = 0; 0,01; \cdots; 1
\]

\item 
\[
 \int_0^1 (\alpha x+1)^{\sin x}\,\mathrm{d}x, \text{čia } \alpha = 0; 0,01; \cdots; 1
\]

\item 
\[
 \int_0^1 (1-x)^{\sin x}\,\mathrm{d}x, \text{čia } \alpha = 0; 0,01; \cdots; 1
\]

\item 
\[
 \int_0^1 (1-x)^{\cos x}\,\mathrm{d}x, \text{čia } \alpha = 0; 0,01; \cdots; 1
\]

\item 
\[
 \int_0^1 (1-x)^{\ln (x+1) }\,\mathrm{d}x, \text{čia } \alpha = 0; 0,01; \cdots; 1
\]

\item 
\[
 \int_{-1}^1 (2-x)^{(\alpha + x)}\,\mathrm{d}x, \text{čia } \alpha = 0; 0,01; \cdots; 1
\]

\item 
\[
 \int_{-1}^1 (2-x)^{(\alpha x)}\,\mathrm{d}x, \text{čia } \alpha = 0; 0,01; \cdots; 1
\]



\end{enumerate}

\end{document}