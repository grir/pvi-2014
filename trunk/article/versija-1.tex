\documentclass[10pt]{article}
%Paketai----------------------------------------------
 \usepackage[L7x]{fontenc}
 \usepackage[lithuanian]{babel}
 \usepackage[utf8x]{inputenc}
 \usepackage{graphicx}
 \usepackage{latexsym}
 \usepackage{amssymb}
 \usepackage{amsmath}
 \newtheorem{thm}{Teorema}
 \newtheorem{cor}[thm]{Išvada}
 \newtheorem{lem}[thm]{Lema}
 \newtheorem{prop}[thm]{Teiginys}
 \newtheorem{defn}[thm]{Apibrėžimas}
 \newtheorem{rem}[thm]{Pastaba}
 \def\institution#1{{\raggedright #1}}
 \def\address#1{\newline ${\ }$ \emph{#1}}
 \def\emaill#1{{\raggedright el.~paštas:\ {#1}}}
 \def\email#1{\newline {\raggedright ${\qquad\qquad}${#1}}}
 \def\dedication#1{\vspace{2mm}{\raggedright\qquad\emph{Dedikuotas {#1}}\vspace{2mm}}}
 \def\abstract#1{\vspace{2mm}{\raggedright\textbf{Santrauka}. {#1}\vspace{1mm}}}
 \def\keywords#1{{\raggedright\emph{Raktiniai žodžiai}: {#1}}}
 \def\Summary#1#2#3#4{\vspace{2mm}{\textbf{Summary}}\vspace{2mm}\newline{\textbf{#1}}\newline{\emph{#2}}\newline{#3}\newline{{\emph{Keywords:\ }}#4}}
 \def\boldhline{\noalign{\global\arrayrulewidth.8pt}\hline\noalign{\global\arrayrulewidth.4pt}}
% papildomi Paketai----------------------------------------------
 \usepackage{color}
% TITLE --------------------------------------------------
%\title{LMR teikiamas straipsnis\thanks{pavyzdys}}
\title{Kai kurie polivalentinių sistemų modeliavimo geometriniai aspektai}

\author{Irus Grinis${}^{1}$, Feliksas Ivanauskas${}^{1}$, Gediminas Stepanauskas${}^{1}$}
\date{ }
%literatūros sąrašas======================================================
\begin{filecontents*}{x.bib}

@BOOK{latex,
   author = "Leslie Lamport",
   title = "{\LaTeX:} {A} Document Preparation System",
   publisher = "Addison-Wesley",
   year = 1986
}

@BOOK{knygaLt,
 author={P.P. Autorius and A. Autorius and T. Autorius},
 title={ Knygos pavadinimas},
 publisher={Leidykla},
 address={Adresas},
 year={2009}
}

@BOOK{knygaEn,
 author={F.F. Author and S. Author and T. Author},
 title={ Book title},
 publisher={Publisher},
 address={Address},
 year={2009}
}

@ARTICLE{StraipsnisLt,
 author={P.P. Autorius and A. Autorius and T. Autorius},
 title={Straipsnio pavadinimas},
 journal={Žurnalas},
 volume={75},
 number={1},
 year={1995},
 pages={15-31},
 note={Pastaba}
}

@ARTICLE{StraipsnisEn,
 author={F.F. Author and S. Author and T. Author},
 title={Title of Article},
 journal={Journal},
 volume={75},
 number={1},
 year={1995},
 pages={15-31},
 note={Note}
}

@INPROCEEDINGS{inproceedingsEN,
   author = {F.F. Author and S. Author and T. Author},
   title = {Article in proceedings},
   editor = {C. Editor and S. Editor and S.A. Editor},
   booktitle = {Proc. of the  Intern. Conference, City, Country, 2009},
   series = {Series},
   pages = {1--15},
   year = {2009},
   address = {Address},
   publisher = {Publisher}
 }

@INPROCEEDINGS{KonfLeidLT,
   author = {P.P. Autorius and A. Autorius and T. Autorius},
   title = {Straipsnis konferencijos darbuose},
   editor = {C. Redaktorius and S. Redaktorius and S.A. Redaktorius},
   booktitle = {Konferencijos darbai, Miestas, Šalis, 2009},
   series = {Serija},
   pages = {1--15},
   year = {2009},
   address = {Adresas},
   publisher = {Leidykla}
 }
\end{filecontents*}
%======================================================

%%% --------------------------------------------------------
\begin{document}
\maketitle
\vspace{-0.5cm}
{\small



\institution{${}^{1}$Vilniaus Universitetas, Matematikos ir informatikos fakultetas}
\address{Naugarduko g. 24, LT-03225 Vilnius, Lietuva}

\vspace{2mm}
\emaill{irus.grinis@mif.vu.lt; feliksas.ivanauskas@mif.vu.lt;}
\email{gediminas.stepanauskas@ktl.mii.lt}

%\dedication{}
}

\abstract{ Šiame darbe naginėjami kai kurie polivalentinių sistemų modeliavimo geometriniai aspektai:  
skaičiuojamos valentingumo skirstiniai įvairiems plokštumos regionams }

\keywords{ polivalentinės sąveikos, receptoriai, ligandai, polivalentinės sistemos, matematinis modeliavimas  }
% ----------------------------------------------------------

%\nocite{*}
\section*{Įvadas}
Mažos biologinės dalelės (atskiros molekulės, baltymo, DNR, ląstelės membranos, virusų, bakterijų ar pan. ) valentingumas – atskirų tos pačios rūšies jungčių su kita dalele skaičius (žr., pavyzdžiui, [1], dėl terminologijos). Atskira  jungtis (toliau ją vadinsime tiesiog ryšiu)   tarp dalelių formuojama sąveikos tarp ligando ir receptoriaus pagalba. Sąveikos, kurių valentingumas didesnis už vienetą, vadinamos polivalentinėmis. Mes nagrinėjame kai kuriuos geometrinius  polivalentinių sąveikų modeliavimo aspektus. Valentingumas priklauso nuo daugelio fizikinių, cheminių, sąveikaujančiųjų dalelių geometrinių  savybių. Atskirų polivalentinų sistemų analizei skirta  nemažai dėmesio ([1], [5],[6]) Šiame darbe  mes pateikiame ligando ir receptoriaus, receptorių paviršiaus, ligandų  komplekso abstrakčius  matematinius modelius, kurie gali būti panaudojami  kai kurių polivalentinių sąveikų analizėje. Po to mes teoriškai ir skaitmeniškai  įvertiname valentingumo skirstinius įvairiems  receptorių paviršių ir ligandų kompleksų atvejams

%Straipsniai teikiami surinkti \LaTeX formate naudojant standartinį \emph{Article} stilių.
%Recenzavimui pakanka pateikti straipsnio elektroninę versiją \textbf{PDF} formatu.%%
%Redkolegijos priimtą spaudai straipsnį reikia pateikti elektroninėje versijoje
%(\textbf{\LaTeX} failas su atliktais autoriaus taisymais ir jį atitinkantis \textbf{PDF}
%failas bei visų pavekslėlių  \textbf{EPS} failai).

\section{Apibrėžimai, matematiniai modeliai, elementarūs tyrinėjimai}

\subsection{Apibrėžimai, modelių aprašai}
Tarkime, kad turime dviejų rūšių sąveikaujančias daleles. Viena rūšis (pavyzdžiui, bakterija, organizmo ląstelė) turi savo paviršiuje ar jo dalyje tam tikrą skaičių aktyvių vietų, kuriuos  vadinsime receptoriais. Matematiškai galima asocijuoti minėtą paviršių su tam tikru geometriniu paviršiumi, o atskirą receptorių -  su tašku tame paviršiuje. Ligandai  sąveikauja su receptoriais pagal taisyklę vienas ligandas – vienas receptorius.  Ligandų ir receptrių vietą erdvėje asocijuosime su taškų, kuriuos vadinsime ligando ar receptoriaus centrais, koordinatėmis. Šiame darbe sąveiką mes apibrėžiame kaip tam tikrą neneigiamą skaičių, kuris vadinamas sąveikos spinduliu. Jeigu atstumas tarp ligando ir receptoriaus centrų mažesnis, negu nurodytas sąveikos spindulys, tai tarp jų atsiranda ryšys. Ryšio centru vadinsime atitinkamo receptoriaus centro koordinates.

Ligandus ir receptorius mes žymėsime atitinkamai  mažosiomis  $l$ ir $r$ raidėmis (su indeksais, jeigu reikia). Šiame darbe mes nagrinėjame atvejį, kai viena iš dviejų sąveikaujančių dalelių yra pakankamai didelė, kad tam tikrą dalį jos paviršiaus mes galėtume aproksimuoti plokštuma, kurioje ji turi  pagal tam tikrą dėsnį išdėstytų  receptorius. Kita dalelė – ligandų kompleksas –  aprėžta plokštuminė sritis, kurioje yra N  ligandų. Ignoruodami daugybę fizikinių  reiškinių, nagrinėjame paprastą  sąveikos  modelį: 
       1)   ligandų kompleksas ,,krenta'' atsitiktinėje receptorių plokštumos vietoje, tolimesniam ryšių formavimui turi įtakos tik atstumai plokštumoje;
        	2)    iš visų $N$ ligandų pirmas suformuoja ryšį tas, kurio atstumas iki artimiausio receptoriaus yra  mažiausias ir neviršija sąveikos spindulio;
            	3)  analogiškai antras ryšys atsiras tarp to ligando ir receptoriaus, tarp kurių atstumas mažiausias iš likusių ir t.t.  
          Bendras susiformavusių pagal šią schemą ryšių skaičius  ir bus valentingumas. Modelis, kuris tenkina išvardintas savybes, vadinamas modeliu  be adaptacinės sąveikos. Paveikslėlyje 2 iliustruojama dvivalenčio ligandų komplekso sąveika su receptoriais, kurios metu susiformuoja  abu ryšiai. Jeigu ligandų kompleksas po pirmo ryšio susiformavimo pradeda transformuotis tol, kol nebus galimybės susiformuoti dar vienam ryšiui, tai tokią sąveiką vadinama  adaptacine.  Toliau nagrinėsime plokštumą su receptoriais, kurių centrai išsidėstę nepriklausomai vienas nuo kito pagal Puasono dėsnį (žr., pavyzdžiui, [4]) su intensyvumu $\lambda$, visas sąveikas laikysime neadaptacinėmis. 
%Visi brėžiniai pateikiami atskirai \textbf{EPS} formatu.

\begin{figure}[t]
\centering {
\begin{minipage}[t]{7.4cm}
{\includegraphics[ scale=0.7]{pav-1.eps}}
\caption{Receptorių plokštuma ir ligandų kompleksas }\label{fig:d1a}
\end{minipage}
%\begin{minipage}[t]{3.7cm}
%{\includegraphics[ scale=0.45]{guide3.eps}}
%\caption{ 3 paveikslėlis.}\label{fig:d1c}
%\end{minipage}
}
\end{figure}


\begin{figure}[t]
\centering {
\begin{minipage}[t]{7.4cm}
{\includegraphics[ scale=0.7]{pav-2.eps}}
\caption{Dvivalenčio ligandų komplekso ir receptorių plokštumos sąveika. Tam, kad susiformuotų abu ryšiai, reikia, kad abiejų ligandų centrai plokštumoje būtų nutolę nuo receptorių atstumu neviršijančiu $R$}\label{fig:d1b}
\end{minipage}
\
%\begin{minipage}[t]{3.7cm}
%{\includegraphics[ scale=0.45]{guide3.eps}}
%\caption{ 3 paveikslėlis.}\label{fig:d1c}
%\end{minipage}
}
\end{figure}




%\begin{figure}[t]
%\centering {
%\begin{minipage}[t]{4.1cm}
%{\includegraphics[ scale=0.5]{guide1.eps}}

%\centerline{a)}
%\end{minipage}
%\quad
%\begin{minipage}[t]{4.1cm}
%{\includegraphics[ scale=0.5]{guide2.eps}}

%\centerline{b)}
%\end{minipage}
%}
%\caption{ Du paveikslėliai.}\label{fig:d2}
%\end{figure}

%\begin{figure}[t]
%\centering {
%\begin{tabular}{ccc}
%{\includegraphics[ scale=0.38]{guide1.eps}}&
%{\includegraphics[ scale=0.38]{guide2.eps}}&
%{\includegraphics[ scale=0.38]{guide3.eps}}\\
%guide1.eps & guide2.eps & guide3.eps \\
% & & \\
%{\includegraphics[ scale=0.38]{guide2.eps}}&
%{\includegraphics[ scale=0.38]{guide1.eps}}&
%{\includegraphics[ scale=0.38]{guide2.eps}}\\
%guide2.eps& guide1.eps & guide2.eps \\
%\end{tabular}
%\caption{ Šeši paveikslėliai lentelėje. }\label{fig:d3}
%}
%\end{figure}


%\begin{figure}[t]
%\centering {
%\begin{minipage}[t]{3.6cm}
%{\includegraphics[ scale=0.45]{guide1.eps}}

%\centerline{1}
%\end{minipage}
%\quad
%\begin{minipage}[t]{3.6cm}
%{\includegraphics[ scale=0.45]{guide1.eps}}

%\centerline{2}
%\end{minipage}
%\quad
%\begin{minipage}[t]{3.6cm}
%{\includegraphics[ scale=0.45]{guide1.eps}}

%\centerline{3}
%\end{minipage}
%\caption{Ta pati figūra. }\label{fig:d5} }
%\end{figure}

%\begin{figure}[t]
%\centering {
%\begin{minipage}[t]{10cm}
%{\includegraphics[ scale=0.6]{guide4.eps}} \caption{Vienas paveikslėlis.}\label{fig:d1g}
%\end{minipage}
%}
%\end{figure}

%Paveikslėliams įdėti naudojama standartinė \LaTeX\ {\color{green}\verb!figure!}
%({\color{green}{\tt graphicx} paketas}) aplinka. Visiems EPS failams naudokite tą patį
%vardą, t.y. {\color{magenta}{\tt guide.tex, guide1.eps,..., guide4.eps}}. Šio pavyzdžio
%paveikslėlių originalūs išmatavimai:
%\begin{quote}
%\item[1)] {\color{magenta}{\tt guide1.eps}} --  8cm x 8cm,
%\item[2)] {\color{magenta}{\tt guide2.eps}} --  8cm x 8cm,
%\item[3)] {\color{magenta}{\tt guide3.eps}} --  8cm x 8cm,
%\item[4)] {\color{magenta}{\tt guide4.eps}} -- 16cm x 8cm.
%\end{quote}

%Pateikti pavyzdžiai:  1) trys atskiri paveikslėliai \ref{fig:d1a}~pav., \ref{fig:d1b}~pav.,\ref{fig:d1c}~pav.;
% 2) 2 susieti paveikslėliai \ref{fig:d2}~pav.;
% 3) paveikslėliai lentelėje \ref{fig:d3}~pav.; 4) tie patys paveikslėliai  \ref{fig:d5}~pav.;
% 5) vienas didelis paveikslėlis  \ref{fig:d1g}~pav.



%Pastaba. Paveikslėlių failai priimami tik {\quotedblbase}encapsulated PostScript{\textquotedblleft} formatu
%{\color{magenta}{\tt *.EPS }}!

\subsection{Elementarūs tyrinėjimai}

\begin{thm}[1]\label{thm:1}
Tegul receptorių centrai išsidėstę plokštumoje  pagal Puasono dėsnį su intensyvumu $\lambda$, o sąveikos spindulys yra $R$. Tada teisingi teiginiai:
\begin{enumerate}
	\item tikimybė, kad vienvalentinis ligando kompleksas susiformuos ryšį su kokiu nors receptoriumi plokštumoje yra $1-\exp(-\lambda \pi R^{2})$;
	\item jeigu $N$ ligandų kompleksas yra toks, kad tarp bet kokių dviejų ligandų atstumas  yra didesnis už $2R$, tai, tikimybė, kad susiformuos  visi $N$ ryšiai  yra $(1-\exp(-\lambda \pi R^{2}))^N$;
	\item ankstesnio punkto  sąlygomis  tikimybė, kad kompleksas  sudarys bent vieną ryšį yra 
	$1-\exp(-N \lambda \pi R^{2}))$;
	\item antro punkto sąlygomis tkimybė, kad kompleksas susformuos lygiai $k$ ryšių yra 
	 $ \binom{N}{k} \left( 1-\exp(-\lambda \pi R^{2}) \right) ^ k \exp(-(N-k) \lambda \pi R^{2})  $ 
	
	
	
%	\item tikimybė, kad vienvalentinis ligando kompleksas susiformuos ryšį su kokiu nors receptoriumi plokštumoje yra $1 - \exp(- \lambda \cdot \pi \cdot \R^{2})$;	 
\end{enumerate}

\end{thm}



%\begin{table}[b]
%\caption{Taip atrodo trijų stulpelių paprasta lentelė.}\label{t1}
% \centering
% \tabcolsep=5pt
% \vspace{2mm}
%\begin{tabular}{rrr}
%\boldhline
%&  $a$ & $b$ \\
%\hline
%$x$ & 1.12 & 0.11\\
%$y$ & 10.34 & 0.2\\
%\boldhline
%\end{tabular}
%\end{table}
%\begin{table}[t]
%\caption{Sudėtingesnės lentelės pavyzdys.}\label{t2}
% \centering
% \tabcolsep=5pt
% \vspace{2mm}
%%\raggedright \tabcolsep=5pt
%\begin{tabular}{lcccccccccc}
%\boldhline %\\[-1pt]
%& \multicolumn{3}{c}{$f(x)$} && \multicolumn{3}{c}{$g(x)$}
%&& \multicolumn{2}{c}{Metodas} \\[2pt] %
%\cline{2-4}\cline{6-8}\cline{10-11}
%\multicolumn{11}{l}{}\\[-7pt]
%& $\infty$ & \d{3}2 & \d{1}1 && $\infty$ & \d{1}2 & \d{1}1
%&& Standartinis & Modifikuotas \\
%A & $\infty$ & 2105 & nėra &&  269  & 65  &
% 22 && $\infty$ & 10 \\
%B & 46 & \d{2}46 & 47 && \d{2}5 & \d{1}5 & \d{1}5
%&& 23 & \d{1}5 \\
%\boldhline
%\end{tabular}
%\end{table}

%Naudojama standartinė \LaTeX\ {\color{green}\verb!tabular!} paketo aplinka.
%\ref{t1}~lentelė yra  paprasta, o \ref{t2} lentelė sudėtingesnė.
%
%\section{Formulės}

%Tekste formulės renkamos standartiškai, pvz. tekste $a+b=x^2$;

%necituojamos
%\[{\sf e}^{{\sf i}\pi x}=-1;\]
%cituojamos
%\begin{equation}
%{\sf e}^{{\sf i}\pi x}=-1,\label{eq:2}
%\end{equation}
%prisilaikant standartinių matematinių formulių rinkimo taisyklių.



%\section{Teiginiai ir apibrėžimai}
%
%\begin{thm}[Teoremos pavyzdys]\label{thm:1}
%Teoremos tekstas.
%\end{thm}
%
%\begin{lem}[Lemos pavyzdys]\label{lem:1}
%Lemos tekstas.
%\end{lem}
%
%\begin{cor}[Išvados pavyzdys]\label{cor:1}
%Išvados tekstas.
%\end{cor}
%
%\begin{defn}[Apibrėžimo pavyzdys]\label{defn:1}
%Apibrėžimo tekstas. \emph{Apibrėžiama sąvoka} rašoma kursyvu.
%\end{defn}
%
%\begin{rem}[Pastabos pavyzdys]\label{rem:1}
%Standartiškai apibrėžtos definicijos: teorema (thm), lema (lem), išvada (cor),
%apibrėžimas (defn), pastaba (rem).
%\end{rem}
%
%Jeigu Jums reikaligas kitas tipas, tuomet naudokite teiginio definiciją (prop) nurodydami
%tikrą pavadinimą.
%
%\begin{prop}[Algoritmas]\label{prop:1}
%Čia Algoritmo tekstas. Skliausteliuose parašyta, kad ši dalis yra Algoritmas.
%\end{prop}
%
%\begin{rem}\label{rem:2}
%Standartiniame \emph{Article} stiliuje teiginio ar apibrėžimo numeris eina po teiginio
%{\quotedblbase}\textbf{\emph{Teorema 1}}{\textquotedblleft} (angl. variantas). Teikamas  į LMR žurnalą straipsnio
%variantas yra darbinis, galutiniame straipsnio variante lietuviškuose straipsniuose bus
%{\quotedblbase}\textbf{\emph{1 teorema.}}{\textquotedblleft} pataisyta automatiškai.
%\end{rem}
%
%\section{Citavimas}
%Literatūros šaltiniai
%\cite{StraipsnisEn,inproceedingsEN,knygaEn,StraipsnisLt,knygaLt,KonfLeidLT,latex} renkami
%standartiniame \textbf{PLAIN} stiliuje straipsnio pradžioje. Transliuojant LaTeX ši
%informacija užrašoma į failą {\color{red}x.bib}. Jeigu daromi kokie nors taisymai
%literatūros sąraše, prieš transliuojant LaTeX-u, ši failą reikia pašalinti. Literatūros
%šaltinius reikia būtinai cituoti su {\color{red}\verb!cite!} komanda.
%
%\begin{rem}\label{rem:3}
%Nekreipkite dėmesio į tai, kad lietuviškame straipsnyje literatūros saraše sugeneruojami
%angliški žodžiai. Galutiniame straipsnio variante tai bus pakeista. Autoriaus tikslas yra
%pilnai ir teisingai užpildyti {\color{red}x.bib} dalį failo  pradžioje  savo literatūros
%sąrašo šaltiniais, pasinaudojant šiame faile pateiktais pavyzdžiais.
%\end{rem}
%
%\section{Šablonai ir kalba}
%
%Šablonai \LaTeX failas, šis failas ir jo pdf pateikti keletu variantų:
%
%anglų kalba -- {\color{magenta}{\tt {\color{magenta}{\tt LMD{\textunderscore}en.tex}},
%{\color{magenta}{\tt LMD{\textunderscore}en.pdf}};
%
%lietuvių -- {\color{magenta}{\tt LMD{\textunderscore}gidas.tex}},
%LMD{\textunderscore}gidas.pdf}}, {\color{magenta}{\tt LMD{\textunderscore}lt.tex}},
%{\color{magenta}{\tt LMD{\textunderscore}lt.pdf}}.
%
%\section{\LaTeX failo transliavimas}

%Tarkime Jūsų originalaus failo vardas yra {\color{magenta}name.tex}
%
%Teisingai sukonfiguruotoje \TeX sistemoje (pvz. TeXLive ar MiXTeX) su tvarkingai
%įdiegtais lietuviškais šriftais jokių papildomų failų nereikia.
%
%
%
%\begin{description}\item[]\begin{quote}
%%
%\item[1)] redaguojamas failas  {\color{magenta}name.tex};
%\item[2)] naikinamas {\color{magenta}x.bib};
%\item[3)] vykdoma komanda {\color{red}latex name.tex} .  Gaunamas failas {\color{magenta}x.bib};
%\item[4)] vykdoma komanda {\color{red}bibtex name} .     Gaunamas failas name.bbl;
%\item[5)] vykdoma komanda {\color{red}latex name.tex} .  Gaunamas failas {\color{magenta}name.aux};
%\item[6)] vykdoma komanda {\color{red}latex name.tex} .  Gaunamas failas {\color{magenta}name.dvi};
%\item[7)] vykdoma komanda {\color{red}dvips name.dvi} .  Gaunamas failas {\color{magenta}name.ps};
%\item[8)] {\color{magenta}name.ps} failas konvertuojamas į {\color{magenta}name.pdf};
%\item[9)] jei reikalinga grįžtama  į 1).
%\end{quote}\vspace{0.5ex}\end{description}
%%
%
%\section{Išvados}
%Patartina teikiamą straipsnį rinkti šablone, kurį galima rasti Lietuvos matematikos
%rinkinio interneto puslapyje
%
%{\color{blue}{\tt{http://www.mii.lt/LMR/}}}
%
%Čia galima rasti ir \LaTeX failo rinkimo pavyzdžius. Autoriai pateikia visus savo failus
%per šį internetinį puslapį.
%
%Galutinis straipsnio variantas bus maketuojamas pagal LMR stilių.

% ----------------------------------------------------------
\bibliographystyle{plain}
\bibliography{x}
\Summary{LMR Paper Example}{F. Author, S. Author and T.Author}{Summary text is
Abstract(Santraukos) translation.}{keywords}
\end{document}
