\documentclass[11pt]{beamer}
\usetheme{Copenhagen}
\usepackage[utf8x]{inputenc}
\usepackage[lithuanian]{babel}
\usepackage[L7x]{fontenc}
\usepackage{lmodern}
\usepackage{amsmath}
\usepackage{amsfonts}
\usepackage{amssymb}
\usepackage{easylist}

\author{I.Grinis, F.Ivanauskas, G.Stepanauskas}
\title{Kai kurie polivalentinių sąveikų aspektai}
%\setbeamercovered{transparent} 
%\setbeamertemplate{navigation symbols}{} 
%\logo{} 
\institute{VU MIF} 
\date{2014 06 26} 
\subject{Applied math} 
\begin{document}

\begin{frame}
\titlepage
\end{frame}

\begin{frame}
\tableofcontents
\end{frame}

%%%%%%%%%%%%%%%%%%%%%%%%%%%%%%%%%%%%%%%%%%%%%%%%%%%%%%%%%%%%%%%%%%%%%%%%%%%%

\begin{frame}{Sąvokos}
\begin{itemize}
\item Aplinka -- metrinė erdvė $X$, kurioje vyksta tam tikri procesai.
\item Ligandų kompleksas $\mbox{\textit{Lig}}$ -- sudėtinis objektas, kuris apibūdinamas $N_L$ elementais (taškais)   $l_i \in X, i=1,\cdots, N_L$.
\item Receptorių  kompleksas $\mbox{\textit{Rec}}$ -- sudėtinis objektas, kuris apibūdinamas $N_R$ elementais (taškais) $ r_i \in X, i=1, \cdots, N_R$. 
\item Laikas -- parametras, nuo kurio priklauso $\mbox{\textit{Lig}}$ ir $\mbox{\textit{Rec}}$. Paprastai žymimas $t>0$, $t \in \mathbb{R}$ arba $t$, $t \in \mathbb{Z}$.
\item Ligandų ir receptorių kompleksai gali \textit{sąveikauti}.
\end{itemize}
\end{frame}

%%%%%%%%%%%%%%%%%%%%%%%%%%%%%%%%%%%%%%%%%%%%%%%%%%%%%%%%%%%%%%%%%%%%%%%%%%%%

\begin{frame}{Ligando kompleksas}
\begin{itemize}
\item Ligandų komplekso taškai charakterizuojami tarpusavyje poriniais atstumais.
\item Ligandų komplekso energija $E_L$  -- funkcija nuo minėtų atstumų.
    \begin{itemize}
       \item Pvz.: $E_L = \sum\limits_{i,j=1,i<j}^{N_L} k_{ij}^L (d(l_i,l_j) - d_{ij}^L)^{2} $
    \end{itemize}
\item Postuluojama, kad jeigu kompleksas nesąveikauja su receptorių kompleksu, tai $E_L = 0$.
\end{itemize}
\end{frame}

%%%%%%%%%%%%%%%%%%%%%%%%%%%%%%%%%%%%%%%%%%%%%%%%%%%%%%%%%%%%%%%%%%%%%%%%%%%%

\begin{frame}{Receptorių kompleksas}
\begin{itemize}
\item Receptorių komplekso taškai taip pat charakterizuojami tarpusavyje poriniais atstumais.
\item Receptorių komplekso energija $E_R$  -- funkcija nuo minėtų atstumų.
    \begin{itemize}
       \item Pvz.: $E_R = \sum\limits_{i,j=1,i<j}^{N_R} k_{ij}^R (d(r_i,r_j) - d_{ij}^R)^{2} $
    \end{itemize}
\item Postuluojama, kad jeigu kompleksas nesąveikauja su receptorių kompleksu, tai $E_R = 0$.
\end{itemize}
\end{frame}


%%%%%%%%%%%%%%%%%%%%%%%%%%%%%%%%%%%%%%%%%%%%%%%%%%%%%%%%%%%%%%%%%%%%%%%%%%%%

\begin{frame}{Receptorių ir ligandų sąveika }
\begin{itemize}
\item Nagrinėjant ligandų ir receptorių kompleksų evoliuciją, t.y. atitinkamų atstumų priklausomybę nuo laiko, 
svarbu atsižvelgi į galimą jų \textit{sąveiką}, kuri apibūdinama atsirandančiais \textit{ryšiais} tarp atskirų ligandų ir receptorių.
\item Receptorių ir ligando kompleksų taškai $r_i$ ir $l_j$ gali sudaryti \textit{ryšį}, jeigu  atstumas tarp jų 
$d(r_i,l_j) \leqslant R$, kur $R$ -- tam tikras teigiamas realus skaičius. 
\item Minėtas ryšis pasižymi tam tikra \textit{energija} $E_s$, kuri laikoma mažesne už nulį.
\item Jeigu ligandų ir receptorių kompleksai  sąveikauja taip, kad tarp jų atsirado $K$ ryšių, tai bendra sąveikos energija
yra $E = E_R + E_L + K \times E_s$
\item Minėtas skaičius $K$ vadinamas kompleksų ryšio \textit{valentingumu}.
\end{itemize}
\end{frame}

%%%%%%%%%%%%%%%%%%%%%%%%%%%%%%%%%%%%%%%%%%%%%%%%%%%%%%%%%%%%%%%%%%%%%%%%%%%%

\begin{frame}{Receptorių ir ligandų sąveika }
\begin{itemize}
\item Nagrinėjant ligandų ir receptorių kompleksų evoliuciją, t.y. atitinkamų atstumų priklausomybę nuo laiko, 
svarbu atsižvelgi į galimą jų \textit{sąveiką}, kuri apibūdinama atsirandančiais \textit{ryšiais} tarp atskirų ligandų ir receptorių.
\item Receptorių ir ligando kompleksų taškai $r_i$ ir $l_j$ gali sudaryti \textit{ryšį}, jeigu  atstumas tarp jų 
$d(r_i,l_j) \leqslant R$, kur $R$ -- tam tikras teigiamas realus skaičius. 
\item Minėtas ryšis pasižymi tam tikra \textit{energija} $E_s$, kuri laikoma mažesne už nulį.
\item Jeigu ligandų ir receptorių kompleksai  sąveikauja taip, kad tarp jų atsirado $K$ ryšių, tai bendra sąveikos energija
yra $E = E_R + E_L + K \times E_s$
\item Minėtas skaičius $K$ vadinamas kompleksų ryšio \textit{valentingumu}.
\end{itemize}
\end{frame}


%%%%%%%%%%%%%%%%%%%%%%%%%%%%%%%%%%%%%%%%%%%%%%%%%%%%%%%%%%%%%%%%%%%%%%%%%%%%

\begin{frame}{Receptorių ir ligandų sąveika }
\begin{itemize}
\item Nagrinėjant ligandų ir receptorių kompleksų evoliuciją, t.y. atitinkamų atstumų priklausomybę nuo laiko, 
svarbu atsižvelgi į galimą jų \textit{sąveiką}, kuri apibūdinama atsirandančiais \textit{ryšiais} tarp atskirų ligandų ir receptorių.
\item Receptorių ir ligando kompleksų taškai $r_i$ ir $l_j$ gali sudaryti \textit{ryšį}, jeigu  atstumas tarp jų 
$d(r_i,l_j) \leqslant R$, kur $R$ -- tam tikras teigiamas realus skaičius. 
\item Minėtas ryšis pasižymi tam tikra \textit{energija} $E_s$, kuri laikoma mažesne už nulį.
\item Jeigu ligandų ir receptorių kompleksai  sąveikauja taip, kad tarp jų atsirado $K$ ryšių, tai bendra sąveikos energija
yra $E = E_R + E_L + K \times E_s$
\item Minėtas skaičius $K$ vadinamas kompleksų ryšio \textit{valentingumu}.
\end{itemize}
\end{frame}




%%%%%%%%%%%%%%%%%%%%%%%%%%%%%%%%%%%%%%%%%%%%%%%%%%%%%%%%%%%%%%%%%%%%%%%%%%%%

\begin{frame}{Uždaviniai}
\begin{itemize}
\item Duotiems \textit{Lig} ir \textit{Rec} kompleksams ir žinomiems atstumams $d(r_i, l_j)$ rasti valentingumą, kuriam esant bendra sąveikos energija minimali.

\end{itemize}
\end{frame}




\end{document}